%==============================================================================
% Sjabloon onderzoeksvoorstel bachproef
%==============================================================================
% Gebaseerd op document class `hogent-article'
% zie <https://github.com/HoGentTIN/latex-hogent-article>

% Voor een voorstel in het Engels: voeg de documentclass-optie [english] toe.
% Let op: kan enkel na toestemming van de bachelorproefcoördinator!
\documentclass{hogent-article}

% Invoegen bibliografiebestand
\addbibresource{references.bib}

% Informatie over de opleiding, het vak en soort opdracht
\studyprogramme{Professionele bachelor toegepaste informatica}
\course{Bachelorproef}
\assignmenttype{Onderzoeksvoorstel}
% Voor een voorstel in het Engels, haal de volgende 3 regels uit commentaar
% \studyprogramme{Bachelor of applied information technology}
% \course{Bachelor thesis}
% \assignmenttype{Research proposal}

\academicyear{2025-2026} % TODO: pas het academiejaar aan

% TODO: Werktitel
\title{Hoe kan een webapplicatie ontworpen worden die real-time voertuigdata performant kan simuleren, streamen en visualiseren, op basis van OBD-II-parameters, en die later naadloos kan worden uitgebreid naar echte live datastromen?}

% TODO: Studentnaam en emailadres invullen
\author{Dalschaert Milan}
\email{dalschaert.milan@student.hogent.be}

% TODO: Medestudent
% Gaat het om een bachelorproef in samenwerking met een student in een andere
% opleiding? Geef dan de naam en emailadres hier
% \author{Yasmine Alaoui (naam opleiding)}
% \email{yasmine.alaoui@student.hogent.be}

% TODO: Geef de co-promotor op
\supervisor[Co-promotor]{/}

% Binnen welke specialisatierichting uit 3TI situeert dit onderzoek zich?
% Kies uit deze lijst:
%
% - Mobile \& Enterprise development
% - AI \& Data Engineering
% - Functional \& Business Analysis
% - System \& Network Administrator
% - Mainframe Expert
% - Als het onderzoek niet past binnen een van deze domeinen specifieer je deze
%   zelf
%
\specialisation{Mobile \& Enterprise development}
\keywords{Real-time vehicle data, OBD-II, data simulation, performance, web applications, visualization}

\begin{document}

\begin{abstract}
  Dit voorstel onderzoekt hoe real-time voertuigdata op een performante en overzichtelijke manier kan worden verwerkt en gevisualiseerd binnen een webapplicatie, gericht op autoliefhebbers en technisch geïnteresseerde bestuurders die meer inzicht willen krijgen in de prestaties van hun voertuig dan wat standaard dashboards bieden. De probleemstelling vertrekt vanuit de beperkte toegang tot detailrijke telemetriedata en de nood aan gebruiksvriendelijke toepassingen die continu veranderende informatie kunnen tonen zonder prestatieverlies. De centrale onderzoeksvraag luidt: Hoe kan een webapplicatie ontworpen worden die real-time voertuigdata performant kan simuleren, streamen en visualiseren, op basis van OBD-II-parameters, en die later naadloos kan worden uitgebreid naar echte live datastromen? Het doel van de bachelorproef is een technisch onderbouwd prototype dat OBD-II-gegevens gebruikt als uitgangspunt voor de simulatie, deze data in real-time verwerkt en op een performant manier visualiseert, en aantoont hoe de ontwikkelde architectuur kan dienen als basis voor integratie met echte voertuigtelemetrie. Methodologisch bestaat het onderzoek uit literatuurstudie rond OBD-II, real-time datastreaming, webperformantie en visualisatietechnieken, gecombineerd met ontwerp, implementatie en evaluatie van een webgebaseerde datastreaming- en visualisatieoplossing. Verwacht wordt dat het prototype inzicht geeft in welke technologieën en architecturen het meest geschikt zijn voor performante real-time dataverwerking en hoe gesimuleerde data later probleemloos vervangen kan worden door echte voertuigdata. De meerwaarde van dit onderzoek ligt in het aanbieden van een schaalbare en toegankelijke oplossing voor gebruikers die hun voertuig beter willen begrijpen, evenals een technische blauwdruk voor verdere ontwikkeling in automotive-, IoT- of fleet-managementtoepassingen.
\end{abstract}

\tableofcontents

% De hoofdtekst van het voorstel zit in een apart bestand, zodat het makkelijk
% kan opgenomen worden in de bijlagen van de bachelorproef zelf.
%---------- Inleiding ---------------------------------------------------------

% TODO: Is dit voorstel gebaseerd op een paper van Research Methods die je
% vorig jaar hebt ingediend? Heb je daarbij eventueel samengewerkt met een
% andere student?
% Zo ja, haal dan de tekst hieronder uit commentaar en pas aan.

%\paragraph{Opmerking}

% Dit voorstel is gebaseerd op het onderzoeksvoorstel dat werd geschreven in het
% kader van het vak Research Methods dat ik (vorig/dit) academiejaar heb
% uitgewerkt (met medesturent VOORNAAM NAAM als mede-auteur).
% 

\section{Inleiding}%
\label{sec:inleiding}

Moderne voertuigen genereren enorme hoeveelheden data die het potentieel hebben om bestuurders en eigenaars veel dieper inzicht te geven in de prestaties en gezondheid van hun wagen. Toch blijft deze informatie vaak versnipperd, moeilijk toegankelijk of technisch complex om te interpreteren. In deze bachelorproef richt ik mij op het ontwikkelen van een oplossing die deze technische voertuigdata op een duidelijke en performante manier visualiseert, specifiek voor autoliefhebbers en technisch geïnteresseerde autobezitters die hun auto beter willen begrijpen en nauwkeuriger willen opvolgen.

Het thema wordt gekaderd binnen de bredere evolutie naar data-gedreven voertuigen, waarbij real-time gegevens zoals motortoerental, vermogen, brandstofverbruik of sensordiagnoses steeds vaker beschikbaar zijn via OBD-interfaces, API’s of telematica-systemen. De uitdaging bestaat er echter in om deze data zo te verwerken en te presenteren dat gebruikers er daadwerkelijk inzichten uit kunnen halen, zonder nood aan diepgaande technische of programmeerkennis.

De concrete probleemstelling luidt als volgt: hoe kan voertuigdata in real-time performant en begrijpbaar worden gevisualiseerd voor autoliefhebbers die hun voertuig nauwkeuriger willen monitoren en analyseren? Vanuit deze probleemstelling vertrekt de centrale onderzoeksvraag:
“Welke technologieën en visualisatiestrategieën maken het mogelijk om real-time voertuigdata performant, stabiel en gebruiksvriendelijk weer te geven voor technisch geïnteresseerde autobezitters?”

De doelstelling van het onderzoek is tweeledig. Enerzijds wil ik een duidelijk en onderbouwd inzicht bieden in welke technologieën (zoals data-streaming, caching, frameworks voor visualisatie) het meest geschikt zijn voor dit type toepassing. Anderzijds werk ik toe naar een concreet proof-of-concept: een webapplicatie die real-time voertuigdata verwerkt en visualiseert op een manier die aansluit bij de noden van autoliefhebbers. Het eindresultaat moet aantonen dat de gekozen technieken effectief leiden tot een vlotte, begrijpbare en betrouwbare gebruikerservaring.

%---------- Stand van zaken ---------------------------------------------------

\section{Literatuurstudie}%
\label{sec:literatuurstudie}

Het domein van real-time voertuigdata, webgebaseerde visualisaties en OBD-II-diagnostiek bevindt zich op het kruispunt tussen automotive informatics, IoT-architecturen en webperformance. De huidige literatuur toont dat zowel de beschikbaarheid van voertuigdata als de manieren waarop deze data performant verwerkt en gevisualiseerd wordt sterk evolueren. Dit onderzoek sluit aan bij bestaande academische inzichten, maar richt zich op een specifieke open uitdaging: hoe kan real-time voertuigdata op een performante, schaalbare én gebruiksvriendelijke manier beschikbaar worden gesteld aan technisch geïnteresseerde autobezitters, zonder afhankelijk te zijn van complexe hardware of bedrijfsomgevingen.

\subsection{OBD-II als standaard voor voertuigdata}

On-Board Diagnostics (OBD-II) vormt sinds 1996 de universele standaard voor voertuigtelemetrie bij consumentenwagens. De review van \textcite{Michailidis2025} toont aan dat OBD-II een brede reeks parameters beschikbaar maakt, zoals motortoerental, voertuig­ snelheid, koelvloeistoftemperatuur, lucht-brandstofverhouding en foutcodes (DTC’s). Deze auteurs benadrukken dat OBD-II toegankelijk is, maar tegelijk beperkt in updatefrequentie en afhankelijk van de gebruikte adapter (bijv. ELM327). Deze technische beperkingen zijn relevant voor dit onderzoek, omdat ze bepalen hoe real-time de data werkelijk kan zijn en welke simulatieparameters realistisch zijn.

Daarnaast beschrijven \textcite{Oluwaseyi2020} hoe OBD-II functioneert als digitaal diagnosetool, met een focus op de structuur van PIDs (Parameter IDs) en de uitdagingen rond datakwaliteit en betrouwbaarheid van goedkope adapters. Dit biedt een goede basis voor het deel van dit onderzoek waarin de overgang van gesimuleerde naar echte OBD-II-data wordt onderzocht.

Verder tonen werken zoals \textcite{Dabarera2022} en \textcite{Madushan2023} hoe OBD-II vandaag wordt gecombineerd met IoT-platformen, waarbij data via protocollen zoals MQTT wordt doorgestuurd naar cloudplatformen. Deze studies onderbouwen de haalbaarheid van continue real-time dataflow vanuit voertuigen naar webapplicaties.

\subsection{Real-time data-architecturen en performantie}

Informatie over performante verwerking van real-time data komt vooral uit IoT- en big-dataonderzoek. \textcite{Ferhat2024} beschrijven hoe een voertuigvolgsysteem migratie naar een big-data-omgeving vereist om grote datavolumes efficiënt te verwerken. Hun conclusie is dat traditionele relationele systemen minder geschikt zijn voor real-time data streams, terwijl gedistribueerde systemen met event-driven architecturen betere schaalbaarheid en performantie leveren. Dit is relevant voor de backendarchitectuur van de webapplicatie in deze bachelorproef.

Ook \textcite{Perez-Gonzalez2025} beschrijven een volledig real-time telemetriesysteem voor elektrische racewagens. Hun implementatie toont dat performantie vooral afhankelijk is van:

\begin{itemize}
    \item Sampling rate van sensoren
    \item Datacompressie en transmissiemethode
    \item Efficiënt updaten van dashboards (event-driven i.p.v. polling)
    \item Client-side rendering performance
\end{itemize}

Deze inzichten zijn direct toepasbaar op de performantieanalyse van jouw webapplicatie.

\subsection{Visualisatie van real-time data}

Visualisatie vormt een cruciale component in het interpreteren van telemetrie. \textcite{VandenHautte2020} presenteren een dynamisch dashboarding-framework dat sensordata automatisch detecteert en verwerkt tot real-time visualisaties. Ze benadrukken dat dashboards niet alleen correct moeten zijn, maar vooral responsief, lichtgewicht en geoptimaliseerd voor continue updates.

\subsection{Simulatie van voertuigdata}

Omdat OBD-II-data beperkt is door updatefrequentie en hardwarevariatie, kiezen veel studies voor simulatie als testbasis. In de literatuur rond IoT-telemetrie tonen \textcite{Perez-Gonzalez2025} en \textcite{Madushan2023} dat simulatie cruciaal is om datastromen te valideren nog vóór fysieke sensoren worden aangesloten.

Simulatie wordt vooral ingezet om:
hoge updatefrequenties te testen (bijv. 30–60 updates per seconde),
stress-testing van dashboards te doen,
verschillende voertuigscenario’s (acceleratie, temperatuurverhoging, foutcodes) te reproduceren.

% Voor literatuurverwijzingen zijn er twee belangrijke commando's:
% \autocite{KEY} => (Auteur, jaartal) Gebruik dit als de naam van de auteur
%   geen onderdeel is van de zin.
% \textcite{KEY} => Auteur (jaartal)  Gebruik dit als de auteursnaam wel een
%   functie heeft in de zin (bv. ``Uit onderzoek door Doll & Hill (1954) bleek
%   ...'')

%---------- Methodologie ------------------------------------------------------
\section{Methodologie}%
\label{sec:methodologie}

Het doel van deze bachelorproef is het ontwikkelen van een performante webapplicatie waarin real-time voertuigdata inzichtelijk wordt gemaakt voor autoliefhebbers en technisch geïnteresseerde autobezitters. Om dit te bereiken wordt een gestructureerde aanpak gehanteerd die literatuuronderzoek, simulatie, proof-of-concept (PoC), integratie van echte data en performance-optimalisatie combineert.\\

In de eerste fase wordt een grondige studie uitgevoerd naar de beschikbare OBD-II data en de methodes om deze uit voertuigen uit te lezen. Daarbij wordt onderzocht welke voertuigparameters essentieel zijn, welke technische beperkingen er bestaan bij real-time uitlezing, en hoe toegankelijk deze data is voor autobezitters. Dit vormt de basis voor de simulatie en de verdere ontwikkeling van de webapplicatie. Het resultaat van deze fase is een overzicht van beschikbare data en aanbevelingen voor data-acquisitie.\\

Vervolgens wordt een PoC ontwikkeld waarin voertuigdata gesimuleerd wordt. Het doel van deze simulatie is om een realistisch datamodel te creëren dat gebruikt kan worden in de webapplicatie. Hierdoor kunnen functionaliteit, visualisaties en performance getest worden zonder afhankelijk te zijn van een echt voertuig. De simulatie omvat typische OBD-II parameters zoals snelheid, toerental, motortemperatuur en brandstofverbruik, en wordt gekoppeld aan een webdashboard waarin data real-time wordt weergegeven.\\

In de derde fase wordt onderzocht hoe deze gesimuleerde data kan worden vervangen door echte OBD-II data. Hierbij worden verschillende manieren van data-acquisitie geëvalueerd, zoals kabelverbindingen, Bluetooth of Wi-Fi. De webapplicatie wordt aangepast om deze data in real-time te ontvangen en weer te geven, zodat de overgang van simulatie naar echte voertuigen soepel verloopt.\\

Daarna volgt de fase van performance-optimalisatie. In deze stap worden technieken toegepast om de webapplicatie sneller en stabieler te maken bij real-time data‑streams. Hierbij kan gedacht worden aan caching, throttling van data, gebruik van WebSockets of andere streamingtechnologieën, en optimalisatie van de frontend rendering. Het doel is dat de webapplicatie soepel werkt, ook bij frequente updates van voertuigdata.\\

Tot slot wordt de applicatie geëvalueerd op technische correctheid, prestaties en gebruiksvriendelijkheid. De resultaten worden gedocumenteerd in de bachelorproef, inclusief aanbevelingen voor verdere verbeteringen en uitbreidingen. Het concrete eindresultaat is een werkende webapplicatie die zowel gesimuleerde als echte OBD-II data realtime kan verwerken en visualiseren.

\subsection{Tijdsplanning}

\begin{itemize}
    \item Literatuurstudie en data-analyse (4 weken): overzicht van OBD-II parameters en technische beperkingen.
    \item PoC simulatie (3 weken): simulatie-backend en webdashboard.
    \item Integratie echte OBD-II data (4 weken): werkende data-acquisitie module en integratie in de applicatie.
    \item Performance-optimalisatie (3 weken): geoptimaliseerde realtime webapplicatie en performance rapport.
    \item Evaluatie en documentatie (2 weken): volledige bachelorproef en technische documentatie.
\end{itemize}

Door deze methodologie wordt zowel een concreet technisch resultaat behaald als diepgaande kennis over OBD-II data, real-time dataverwerking en webvisualisatie ontwikkeld, waardoor de bachelorproef voldoende IT-diepte en praktische relevantie krijgt.

%---------- Verwachte resultaten ----------------------------------------------
\section{Verwacht resultaat, conclusie}%
\label{sec:verwachte_resultaten}

\subsection{Verwachte Resultaten en Meerwaarde}
Het voornaamste resultaat van deze bachelorproef is een werkende webapplicatie die real-time voertuigdata kan verwerken en visualiseren, zowel vanuit een simulatie als vanaf echte OBD-II apparaten. De applicatie zal dynamische dashboards bevatten met meerdere grafieken die inzicht geven in verschillende voertuigparameters, zoals snelheid, toerental, brandstofverbruik, motortemperatuur en acceleratie.

\subsection{Verwachte technische resultaten}

\begin{itemize}
    \item Een performante backend die data van meerdere voertuigen tegelijk kan verwerken zonder merkbare vertraging
    \item Realtime updates in de frontend via streamingtechnologieën zoals WebSockets of MQTT
    \item Flexibele dashboards die eenvoudig kunnen worden uitgebreid met nieuwe parameters
    \item Een gestandaardiseerde simulatieomgeving waarmee toekomstige uitbreidingen of andere voertuigen makkelijk kunnen worden getest
\end{itemize}


\subsection{Meerwaarde voor de doelgroep}
Voor autoliefhebbers en technisch geïnteresseerde autobezitters biedt deze bachelorproef verschillende concrete voordelen:

\begin{itemize}
    \item \textbf{Inzicht in voertuigprestaties:} Door real-time data visualisatie krijgen gebruikers een dieper begrip van hoe hun voertuig functioneert en waar verbeteringen mogelijk zijn.
    \item \textbf{Diagnose en preventie:} Onregelmatigheden in motor- of rijgedrag kunnen sneller worden opgespoord, wat preventief onderhoud vergemakkelijkt.
    \item \textbf{Gebruiksvriendelijkheid en toegankelijkheid:} In tegenstelling tot standaard dashboards of fabrikant specifieke tools is de applicatie open, flexibel en gericht op de eindgebruiker die technisch geïnteresseerd is.
    \item \textbf{Basis voor verdere ontwikkeling:} De simulatieomgeving en PoC vormen een fundament voor toekomstige uitbreidingen, zoals koppeling met machine learning-algoritmes voor predictive maintenance of rijgedragsanalyse.
\end{itemize}
.

Kortom, de bachelorproef levert zowel een technisch product op als waardevolle inzichten voor de doelgroep, waardoor zij hun voertuigen beter kunnen begrijpen, bewaken en optimaliseren.



\printbibliography[heading=bibintoc]

\end{document}